\chapter{总结与展望}
\section{本文总结}
近年来,计算机以及智能终端设备的存储和计算能力不断增强,互联网的信息量呈现爆炸式增长。信息量的增加既为人们的生活带来了便捷,也给人们提出了巨大的挑战。据统计,google每天新增的索引网页页数高达40亿。在海量的信息面前如何高效的获取信息以及如何去除冗余信息成了很多人需要面对的问题。

文本是人类数千年历史中主要的信息载体。虽然移动互联网时代的到来大大方便了视频和音频的传播,但是文本依然是人类目前最普遍和高效的信息获取手段。文本匹配作为信息检索和冗余文本消除的基础技术手段,一直受到学术界和工业界的高度重视。

根据使用场景的不同,文本匹配可以被分为三类:短文本-短文本匹配,短文本-长文本匹配以及长文本-长文本匹配。短文本-短文本的匹配主要用于问答系统、问题复述等领域,先得到句子的矩阵表示后,利用深度网络提取句子的语义信息以及句子间的交互信息,将提取出的信息映射为一个概率分布;短文本-长文本的匹配往往需要用到主题模型等辅助,根据长文本的主题分布判断生成短文本的概率;长文本-长文本的匹配相对较少,同样需要利用主题模型等手段将长文本映射到一个高维向量空间,通过高维向量空间的距离衡量两个文本的相似度。

一个优秀的文本匹配算法一般需要解决三个问题:1) 语言的多义性问题,即相同词语在不同环境下的语义问题;2) 语言的组合结构问题,即相同词语的顺序不同导致的语义不同;3) 匹配的非对称性问题,文本匹配任务中两个句子并不需要语义或者结构一致,如问答系统。

为了解决这些问题,近年来有学者试图将深度学习应用于文本匹配任务,取得了巨大的成功。但是这些方法都试图利用深度学习抽取文本中的语义信息,目前对于计算机来说这仍然是不可能的任务,因此基于深度学习的方法具有天然的局限性。

为了解决上述问题,文本利用强化学习抽取匹配过程中句子之间的交互信息。本文从三个角度出发,解决了利用强化学习抽取交互信息的主要难题。首先,本文针对于文本匹配的场景,利用马尔科夫决策过程建模匹配过程中路径的生成过程,利用值迭代方法对马尔科夫决策过程进行训练和预测,验证了马尔科夫决策过程的正确性;其次,利用蒙特卡罗树搜索向前看的特性解决文本匹配的组合结构问题,取得了良好的效果;最后本文对前面设计算法进行了并行实现,提升了算法的运行速度,为大规模分布式训练奠定了基础。
\section{下一步工作}
本文的工作主要集中于利用强化学习解决文本匹配问题。虽然本文提出的算法可以有效地对文本匹配问题做出判别,但是在实验过程中我们依然发现了一些潜在的问题以及其他的相关研究内容。

1. 强化学习的马尔科夫决策过程改进。
本文利用马尔科夫决策过程抽取了文本匹配过程中的两个句子之间的交互信息,但是这只是对文本匹配过程的一种建模方式。后续研究可以对本文提出的马尔科夫过程进行深入的分析以及进一步的改进。

2. 蒙特卡罗树搜索的并行化探索。
为了提高算法的运行速度,本文针对于文本匹配和蒙特卡罗树搜索的场景设计了并行化搜索的算法。该方法虽然可以有效地提高算法的运行速度,但是在和 Tensorflow 交互的处理上仍然有缺陷。因此可以针对本算法的特殊场景,对 Tensorflow 源代码进行修改,以进一步提升算法的运行速度;同时如何做到实时性仍然是蒙特卡罗树搜索的难点所在。
