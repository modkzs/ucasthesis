% \chapter[附录]{附录\quad 中国科学院大学学位论文撰写要求}\chaptermark{附\quad 录}% syntax: \chapter[toc]{title}\chaptermark{header}

% 学位论文是研究生科研工作成果的集中体现,是评判学位申请者学术水平、授予其学位的主要依据,是科研领域重要的文献资料。根据《科学技术报告、学位论文和学术论文的编写格式》(GB/T 7713-1987)、《学位论文编写规则》(GB/T 7713.1-2006)和《文后参考文献著录规则》(GB7714—87)等国家有关标准,结合中国科学院大学(以下简称“国科大”)的实际情况,特制订本规定。

% \textbf{论文无附录者无需此部分}。

\chapter[致谢]{致\quad 谢}\chaptermark{致\quad 谢}% syntax: \chapter[toc]{title}\chaptermark{header}
\thispagestyle{noheaderstyle}% 如果需要移除当前页的页眉
%\pagestyle{noheaderstyle}% 如果需要移除整章的页眉

转眼间又是一个三年,计算所的学习时光是一段特殊的回忆。在这期间我度过了迷 茫彷徨,也经历了斗志昂扬。但是过往即逝,等待我的是新的征程和新的生活。值此论 文即将完成之际,我要向所有关心和支持我的人们致以最真诚的谢意!

特别感谢我的指导老师徐君老师,您治学严谨,循循善诱,精益求 精的品质与精神,令我倍感钦佩,见贤思齐。研究生期间在您指导和帮助下,我 度过了许多学习和科研的难关;您让我有机会参与开发学科前沿项目,让我体会到了 产学结合的魅力,更是让我在为人处世、团队合作和自我修养等方面都有机会提升。您 的言传身教使我终身受益,我将永远铭记在心。

感谢实验室的师兄师姐们:曾玮、崔国歆、庞亮。在我面 临困难的时候,你们总是耐心地帮我解决一个个难题。不仅佩服你们渊博的学识,更折 服于诸位师兄师姐开放包容的风度。

感谢计算所的各位老师和员工,有高屋建瓴的程学旗老师,诲人不倦的刘悦老 师,和蔼可亲的宋铟老师,风趣幽默的崔连军老师,您们无微不至的帮助与支持,让我 们能够顺利地开展项目开发与研究,不断取得新成果。您们就像勤劳的蜜蜂,兢兢业业, 无私奉献,用辛勤的工作陪伴着我们成长。

感谢在研究生三年中相识的所有同学,特别是陈欣洁、李朝辉、于思皓、冯悦、吴晨、王素、吴志达、陶舒畅、丁汉星、曹家硕、黄冬。每个人身上都有着与众不同的闪光点,与你们的友谊是我一 生最美好的回忆。

感谢母校国科大为我们的提供首屈一指的学习生活环境,以及无与伦比的教学资源。

最后,感谢我的父母家人,您们无私无尽的爱是我所有动力的源泉。

\chapter{作者简历及攻读学位期间发表的学术论文与研究成果}
\section*{作者简历}

\begin{center}
姓名:何逸轩 $\qquad$ 性别:男 $\qquad$ 出生日期:1993.11.01 $\qquad$ 籍贯:安徽\\
\end{center}

\noindent
2015.9 -- 2018.7  $\qquad$  中国科学院大学 $\qquad$ 计算技术研究所 $\qquad$  攻读硕士学位

\noindent
2011.9 -- 2015.7   $\qquad$ 北京邮电大学~~~~  $\qquad$ 软件学院~~~~~~~~~~~~ $\qquad$  获得学士学位

\section*{攻读硕士学位期间参加的科研项目} % 有就写,没有就删除
[1] BDA大规模机器学习平台

[2] Tensorflow On Spark

[3] 青岛市智能供热预测信息系统 (青岛泰能热电有限公司)

[4] 微信文本匹配项目

\section*{攻读硕士学位期间的获奖情况} % 有就写,没有就删除
[1] 2017~~中国科学院大学三好学生

[2] 2018~~北纬通信硕士生奖

\cleardoublepage[plain]% 让文档总是结束于偶数页,可根据需要设定页眉页脚样式,如 [noheaderstyle]

